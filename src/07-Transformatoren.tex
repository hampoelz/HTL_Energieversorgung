%\begin{noindent}
\begin{markdown}

# Transformatoren

\GrayBox{Kommt in Energieversorgungsnetzen als Verteil-, Netzkuppel- und Maschinentransformator, als Industrietransformator wie Ofen-, Schweiß-, Stromrichter oder Anlasstransformator oder Bahnstromtransformator vor.}

**Aufgaben von Transformatoren**

- Veränderung der Spannung
- Trennung von Stromkreisen

\end{markdown}

\vspace{1em}

$15 kW$ sollen über eine Entfernung von $500 m$ mit einer Kupferleitung ($\rho = 0,0178 \frac{\Omega mm^2}{m} $) übertragen werden. Berechne die Leitungsverluste für eine Verbraucherspannung von $230 V$ und einem Leitungsquerschnitt von $16mm^2$.

\begin{gather*}
    R = \frac{\rho \cdot l}{A} = \frac{0,0178 \cdot 2 \cdot 500}{16} = 1,11 \Omega \\
    I = \frac{P}{U} = \frac{15000}{230} = 65,22 A \\
    P_V = I^2 \cdot R = 65,22^2 \cdot 1,11 = \underline{4,7 kW}
\end{gather*}

%\begin{noindent}
\begin{markdown}

**Aufbau**

Der aktive Teil besteht aus einem magnetischen Eisenkern mit drei Schenkeln. Die Schenkel tragen zwei oder mehr Drehstromwicklungen. Diese können in Stern, Dreieck oder Zickzack geschalten werden.

Zusätzlich besitzt der Eisenkern noch zwei Rückschlussschenkel, welcher Einfluss auf den Magnetisierungsstrom und die Betriebseigenschaften bei Unsymmetrie nehmen.

\vspace{1em}

**Öltransformator:** \qquad Der aktive Teil befindet sich in Isolier-Öl

- Das Öl verhindert Feuchtigkeitszutritt und führt die Wärme der Wicklung ab.
- Brandgefahr durch ÖL
- Oberfläche wird zur Kühlung groß gehalten, zusätzliche Lüfter werden verwendet.
- Für den Transport wird Öl abgelassen.

\vspace{1em}

**Trockentransformator:** \quad Die Wicklungen sind mit Gießharz vergossen.
\end{markdown}