%\begin{noindent}
\begin{markdown}

# Verbund / Inselbetrieb

\begin{itemize}
    \item[\textbf{Verbundnetz}] entsteht durch verbinden von mehreren kleinen Netzen.
    \item[\textbf{Inselnetz}] ist ein einzelnes (autonom) betriebenes Netz. 
\end{itemize}

## Verbundnetze

\GrayBox{Unabhängig von der Spannungsebene, ist die Struktur des Netzes immer so zu gestalten, dass die Versorgung durch einen Fehler nicht unterbrochen wird. \newline \newline Erst ab zwei Fehler gleichzeitig, kann es zu Versorgungsunterbrechungen kommen.}

**Vorteile:**

\begin{itemize}
    \item \underline{Hohe Versorgungssicherheit}
    \newline Ausfälle eines Kraftwerkes können anhand der großen Anzahl an Kraftwerken leichter bewältigt werden
    \item \underline{Geringere Kosten}
    \newline Schwankungen sind geringer, da aufgrund der großen Anzahl der Verbraucher der kombinierte Strombedarf besser hervorsehbar ist
    \newline Der Bedarf and Reservekraftwerken ist daher geringer
\end{itemize}

Alle großen Stromnetze arbeiten mit Wechselstrom. Daher müssen bei einem Zusammenschluss alle Teilnetze synchronisiert sein:

- Die Spannungsmaxima und Nulldurchgänge müssen zeitgleich erfolgen
- Die Frequenzregelung muss gemeinsam organisiert und umgesetzt sein
- Durch Umrichter und Umformer oder Hochspannungs-Gleichstromübertragung können auch nicht synchronisierte Netze gekoppelt werden


## Inselnetze

**Vorteile**

\begin{itemize}
    \item \underline{Nachhaltige Versorgung}
    \newline Erneuerbare Energien aus Photovoltaik- oder Windenergie können durch steuerbare Stromerzeuger (z.B.: Dieselgeneratoren oder Brennstoffzellen) kombiniert werden
\end{itemize}

**Nachteile**

- **Mehrere Kleinkraftwerke** mit **hoher Zuverlässigkeit** werden benötigt Bei einem allein ist die Ausfallsicherung zu gering
- **Energiespeicher** werden benötigt, um Verbrauchschwankungen auszugleichen
- **Häufige Stromausfälle** Eine hohe Versorgungsqualität erfordert **mehr technischen Aufwand**
- **Verminderte Energieeffizienz** -  Erhöhte Betriebskosten, da verstärkter Teillastbetrieb von z.B. Gasturbienen nötig ist, um Lastschwankungen auszugleichen

\end{markdown}