\section{Asynchronmaschinen}

\GrayBox{Synchronmaschinen werden überwiegend als Motoren verwendet. Als Generator werden sie nur für kleinere Leistungen eingesetzt. (Wasser- oder Windkraftwerken)}

\textbf{Aufbau}

\begin{itemize}
    \item \underline{Feststehender Teil} (Ständer oder Stator)
    \newline Ähnlich aufgebaut wie bei der Synchronmaschine.
    \newline Um den Anlaufstrom zu begrenzen, werden die Drehstromwicklungen mit Schleifringen verbunden. Diese sind kurzgeschlossen und werden während des Anlaufs belastet.
    \item \underline{Beweglicher Teil} (Läufer oder Rotor)
    \newline Besteht aus eine der beiden Wicklungen:
    \subitem Drehstromwicklung
    \subitem Käfigwicklung (Kurzschlussläufer)
\end{itemize}

Im Generator-Betrieb ist die Läuferdrehzahl größer als die synchrone Drehzahl, im Motor-Betrieb umgekehrt.